\documentclass[10pt,a4paper,ragged2e,academicons]{altacv}

\geometry{left=1.25cm,right=1.25cm,top=1.5cm,bottom=1.5cm,columnsep=1.2cm}

\usepackage{nag}
\usepackage{paracol}

\setmainfont{Lato}

\definecolor{Black}{HTML}{000000}
\definecolor{SlateGrey}{HTML}{2E2E2E}
\colorlet{heading}{Black}
\colorlet{accent}{Black}
\colorlet{emphasis}{SlateGrey}
\colorlet{body}{SlateGrey}

\renewcommand{\itemmarker}{\small\faAngleRight}
\renewcommand{\ratingmarker}{\faCircle}

\begin{document}
\name{Michael Kwok}
\personalinfo{
  \email{mkwok1@ualberta.ca}
  \github{n30phyte}
  \linkedin{michael-kwok-2b82a8148}
  \phone{587 937 5980}
}

\AtBeginEnvironment{itemize}{\small}

\makecvheader{}
\columnratio{0.6}
\begin{paracol}{2}
  % PROJECTS
  \cvsection{Education}
  \cvevent{BSc. Computer Engineering}{University of Alberta}{September 2017 -- May 2022}{Edmonton, AB, Canada}

  \cvsection{Work Experience}
  \cvevent{Software Engineer}{Verdi Agriculture}{25 May 2021 -- 30 August 2021}{}
  \begin{itemize}
    \item \textbf{Technologies:} Javascript, Typescript, Jest, MongoDB
  \end{itemize}
  

  \cvevent{Contract App Developer}{AdMore Lighting Inc}{4 August 2020 -- 31 December 2020}{}
  \begin{itemize}
    \item Developed a mobile app as an independent contractor, working closely with full time firmware developer.
    \item Improve existing app's performance and maintainability.
    \item Polish app enough to get released on mobile app stores.
    \item Structure app to allow for automated testing with an external hardware jig in the future.
    \item Fixed over 400 linter warnings, closed over 40 issues.
    \item \textbf{Technologies:} Dart, Flutter, Firebase, Protobuf, Bluetooth Low Energy
  \end{itemize}
  \divider\small

  \cvsection{Projects}
  \cvproject{Rustracer}{15 Aug 2020 -- 16 Aug 2020}{\href{https://github.com/n30phyte/rustracer}{n30phyte/rustracer}}
  \begin{itemize}
    \item A simple raytracer written in Rust completed mostly within 24 hours with small optimizations over the following week.
    \item Got to grips with using Rust for general programming, Rayon for multithreading and AMD uProf for profiling.
    \item \textbf{Technologies:} Rust, Rayon
  \end{itemize}
  \divider\small

  \cvproject{BudgetPacman (Hackathon Project)}{19 Jan 2020}{\href{https://github.com/loravocado/BudgetPacman}{loravocado/BudgetPacman}}
  \begin{itemize}
    \item A project for HackED 2020 with the goal of making a version of tag that used a phone and GPS locations.
    \item Implemented a server using Socket.io and Typescript, learning both technologies overnight.
    \item \textbf{Won:} Best Use of Google Cloud Platform APIs.
    \item \textbf{Technologies:} React Native, Typescript, JavaScript
  \end{itemize}
  \divider\small

  \cvproject{AI Pong (Hackathon Project)}{28 November 2018}{\href{https://github.com/n30phyte/HackEDBeta2018}{n30phyte/HackEDBeta2018}}
  \begin{itemize}
    \item Written during HackED Beta 2018, a hackathon hosted by the University of Alberta Computer Engineering Club.
    \item First non-toy program in C++, using SFML to write a Pong clone to be controlled by a Python AI agent.
    \item \textbf{Technologies:} C++, SFML, Python, Keras, TensorFlow
  \end{itemize}
  \divider\small

  \switchcolumn{}
  \cvsection{Technologies \&  \newline Programming \newline Languages}

  \cvsubsection{Proficient}

  \cvtag{Python}
  \cvtag{Git}
  \cvtag{Java}
  \cvtag{C\#}
  \cvtag{LaTeX}
  \cvtag{CMake}
  \cvtag{Dart}
  \cvtag{C}
  \cvtag{C++}
  \cvtag{CUDA}

  \cvsubsection{Familiar}

  \cvtag{SQL}
  \cvtag{Rust}
  \cvtag{JavaScript}
  \cvtag{VHDL}
  \cvtag{React}
  \cvtag{SFML}
  \cvtag{MongoDB}
  \cvtag{Firebase}
  \cvtag{Ghidra}

  \cvsection{Extra-Curriculars}

  \cvevent{VP Finance}{Indonesian Students' Association Edmonton \\ {\footnotesize PERMIKA Edmonton}}{January 2019 -- December 2020}{}
  \begin{itemize}
    \item Revived the group after 2 years of inactivity due to lack of manpower.
    \item Increased our revenue by 800 dollars within a year.
    \item This funding allowed our group to hold more events, increasing engagement from 5 to 6 people per event to more than 20.
  \end{itemize}
  \divider\small

  \cvevent{Software Team Member}{Autonomous Robotic Vehicle Project}{May 2018 -- December 2019}{}
  \begin{itemize}
    \item Reduce wasted time by writing scripts that help automate repetitive tasks, interfacing with Google App APIs, GitHub’s API and simulation scripts for ROS in C++ and Python.
    \item Improved test coverage for robot’s control systems by writing Google Test unit tests in C++.
    \item Updated technology stack, improving build times and opening access to updated CUDA libraries, increasing performance by 5 times.
  \end{itemize}
  \divider\small
\end{paracol}

\end{document}

