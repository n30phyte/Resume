\documentclass[10pt,a4paper,ragged2e,academicons,withhyper]{altacv}

\geometry{left=1.25cm,right=1.25cm,top=1.5cm,bottom=1.5cm,columnsep=1.2cm}

\usepackage{paracol}

\setmainfont{Lato}

\definecolor{Black}{HTML}{000000}
\definecolor{SlateGrey}{HTML}{2E2E2E}
\colorlet{heading}{Black}
\colorlet{accent}{Black}
\colorlet{emphasis}{SlateGrey}
\colorlet{body}{SlateGrey}

\renewcommand{\itemmarker}{{\small\faAngleRight}}
\renewcommand{\ratingmarker}{\faCircle}

\begin{document}
\name{Michael Kwok}
\personalinfo{
  \email{mkwok1@ualberta.ca}
  \github{n30phyte}
  \linkedin{michael-kwok-2b82a8148}
  \location{Edmonton, AB\kern 1em}
  \phone{587 937 5980\kern 1em}
}

\makecvheader

\AtBeginEnvironment{itemize}{\small}

\columnratio{0.6}

\begin{paracol}{2}

  % PROJECTS
  \cvsection{Education}
  \cvevent{BSc. Computer Engineering}{University of Alberta}{September 2017}{Edmonton, AB, Canada}

  \cvsection{Projects}
  \cvproject{BudgetPacman}{19 Jan 2020}{\href{https://github.com/loravocado/BudgetPacman}{loravocado/BudgetPacman}}
  \begin{itemize}
    \item A project for HackED 2020 with the goal of making a version of tag that used a phone and GPS locations.
    \item Implemented a server using Socket.io and Typescript, learning both technologies overnight.
    \item Won best use of Google Cloud Platform APIs.
    \item \textbf{Technologies:} React Native, Typescript, JavaScript
  \end{itemize}
  \divider\small

  \cvproject{github.io Webpage}{Ongoing Project}{\href{https://github.com/n30phyte/n30phyte.github.io}{n30phyte.github.io}}
  \begin{itemize}
    \item A playground for learning web technologies for single page applications.
    \item \textbf{Technologies:} Vue, HTML, CSS, JavaScript
  \end{itemize}
  \divider\small

  \cvproject{AI Pong}{28 November 2018}{\href{https://github.com/n30phyte/HackEDBeta2018}{n30phyte/HackEDBeta2018}}
  \begin{itemize}
    \item First large scale team project in C++, using SFML to write a Pong clone to be controlled by a Python AI agent.
    \item Written during HackED Beta 2018, a hackathon hosted by the University of Alberta Computer Engineering Club.
    \item Experienced becoming a leader for a small project group,
    \item \textbf{Technologies:} C++, SFML, Python, Keras, TensorFlow
  \end{itemize}
  \divider\small

  \cvproject{SpotifyThing}{2 January 2018}{\href{https://github.com/n30phyte/SpotifyThing}{n30phyte/SpotifyThing}}
  \begin{itemize}
    \item Created a small script to automate the monotonous and repetitive task of turning Spotify playlists to CSV files.
    \item Gained experience in using RESTful Services.
    \item \textbf{Technologies:} C\#, Spotify Web API
  \end{itemize}
  \divider\small

  \nocite{*}

  \switchcolumn

  \cvsection{Programming \newline Languages}

  \cvskill{C++}{4}
  \cvskill{Python}{4}
  \cvskill{C\#}{4}
  \cvskill{\LaTeX}{3}
  \cvskill{Shell Scripting}{3}
  \cvskill{JavaScript}{2}
  \cvskill{TypeScript}{2}
  \cvskill{Verilog}{1}
  \cvskill{Matlab}{1}
  \cvskill{VHDL}{1}

  \cvsection{Technologies \& \newline Programs}

  \cvtag{CMake}
  \cvtag{Google Test}
  \cvtag{Git}
  \cvtag{Vue}
  \cvtag{vcpkg}
  \cvtag{SFML}
  \cvtag{Flutter}
  \cvtag{IntelliJ IDEs}
  \cvtag{Wireshark}
  \cvtag{Visual Studio}
  \cvtag{Visual Studio Code}
  \cvtag{Windows}
  \cvtag{Linux}
  \cvtag{Intel Quartus Prime}
  \cvtag{Xilinx Vivado}

  \cvsection{Volunteering}

  \cvevent{VP Finance}{Indonesian Students' Association Edmonton \\ {\footnotesize PERMIKA Edmonton}}{January 2019 -- Present}{}
  \begin{itemize}
    \item Helped restart the chapter after 2 years of inactivity due to lack of manpower.
    \item Manage finances and budgeting for events and fundraisers.
    \item Successfully budgeted a fundraisers that allowed us to operate comfortably.
    \item Secured funding in the form of grants and sponsorships that allowed our group to grow.
  \end{itemize}
  \divider\small

  \cvevent{Software Team Member}{Autonomous Robotic Vehicle Project}{May 2018 -- December 2019}{}
  \begin{itemize}
    \item Increased productivity by writing scripts that help automate various repetitive tasks, interfacing with Google App APIs, GitHub’s API and simulation scripts for ROS in C++ and Python.
    \item Improved test coverage for robot’s control systems by writing Google Test unit tests.
          %  \item Currently spearheading a rewrite for the robot’s aging simulator to allow for more automated testing, improving simulator stability and reduce reliance on an external unmaintained codebase.
    \item Updated old and undocumented code to fix issues and add features.
  \end{itemize}
  \divider\small

\end{paracol}

\end{document}

