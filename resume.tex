\documentclass[10pt,a4paper,ragged2e,academicons]{altacv}

\geometry{left=1.25cm,right=1.25cm,top=1.5cm,bottom=1.5cm,columnsep=1.2cm}

\usepackage{nag}
\usepackage{paracol}

\setmainfont{Lato}

\definecolor{Black}{HTML}{000000}
\definecolor{SlateGrey}{HTML}{2E2E2E}
\colorlet{heading}{Black}
\colorlet{accent}{Black}
\colorlet{emphasis}{SlateGrey}
\colorlet{body}{SlateGrey}

\renewcommand{\itemmarker}{\small\faAngleRight}
\renewcommand{\ratingmarker}{\faCircle}

\begin{document}
\name{Michael Kwok}
\personalinfo{
  \email{mkwok1@ualberta.ca}
  \github{n30phyte}
  \linkedin{michael-kwok-2b82a8148}
  \location{Edmonton, Alberta}
  \phone{587 937 5980}
}

\AtBeginEnvironment{itemize}{\small}

\makecvheader

\columnratio{0.6}

\begin{paracol}{2}

  % PROJECTS
  \cvsection{Education}
  \cvevent{BSc. Computer Engineering}{University of Alberta}{September 2017}{Edmonton, AB, Canada}

  \cvsection{Work Experience}
  \cvevent{Contract App Developer}{AdMore Lighting Inc}{4 August 2020 -- 31 December 2020}{}
  \begin{itemize}
    \item Worked part time as a developer for an app that controls motorcycle lightbars, including firmware updates and configuration over Bluetooth Low Energy.
    \item Improved modularity by splitting up large widgets, and replacing redundant ones with built-ins.
    \item Improved app performance by removing busy loops and using Dart async features.
    \item App is currently awaiting approval on the Apple App Store.
    \item Replaced unmantained BLE library with a modern one.
    \item \textbf{Technologies:} Dart, Flutter, Firebase, Protobuf, Bluetooth Low Energy
  \end{itemize}
  \divider\small

  \cvsection{Projects}
  \cvproject{Rustracer}{15 Aug 2020 -- 16 Aug 2020}{\href{https://github.com/n30phyte/rustracer}{n30phyte/rustracer}}
  \begin{itemize}
    \item A simple raytracer written in Rust completed mostly within 24 hours with small optimizations over the following week.
    \item Was a vehicle to learn the language while doing something interesting.
    \item Got to grips with using AMD uProf for profiling, Rayon for multithreading and Rust for general programming.
    \item \textbf{Technologies:} Rust, Rayon
  \end{itemize}
  \divider\small

  \cvproject{BudgetPacman (Hackathon Project)}{19 Jan 2020}{\href{https://github.com/loravocado/BudgetPacman}{loravocado/BudgetPacman}}
  \begin{itemize}
    \item A project for HackED 2020 with the goal of making a version of tag that used a phone and GPS locations.
    \item Implemented a server using Socket.io and Typescript, learning both technologies overnight.
    \item \textbf{Won:} Best Use of Google Cloud Platform APIs.
    \item \textbf{Technologies:} React Native, Typescript, JavaScript
  \end{itemize}
  \divider\small

  \cvproject{AI Pong (Hackathon Project)}{28 November 2018}{\href{https://github.com/n30phyte/HackEDBeta2018}{n30phyte/HackEDBeta2018}}
  \begin{itemize}
    \item Written during HackED Beta 2018, a hackathon hosted by the University of Alberta Computer Engineering Club.
    \item First non-toy program in C++, using SFML to write a Pong clone to be controlled by a Python AI agent.
    \item Experienced becoming a leader for programming project.
    \item \textbf{Technologies:} C++, SFML, Python, Keras, TensorFlow
  \end{itemize}
  \divider\small

  \nocite{*}

  \switchcolumn

  \cvsection{Technologies \& \newline Programming \newline Languages}

  \cvsubsection{Proficient}

  \cvtag{C++}
  \cvtag{Python}
  \cvtag{Java}
  \cvtag{C\#}
  \cvtag{\LaTeX}
  \cvtag{SQL}
  \cvtag{CMake}
  \cvtag{Git}
  \cvtag{Flutter}
  \cvtag{Google Test}

  \cvsubsection{Familiar}

  \cvtag{MongoDB}
  \cvtag{C}
  \cvtag{Rust}
  \cvtag{JavaScript}
  \cvtag{VHDL}
  \cvtag{React}
  \cvtag{SFML}
  \cvtag{Firebase}
  \cvtag{Ghidra}

  \cvsubsection{Interests}
  \begin{itemize}
    \item Currently learning about systems programming and backend services.
    \item Computer Graphics and Computer \newline Architecture are some things that I'd like to know more about.
  \end{itemize}

  \cvsection{Volunteering}

  \cvevent{VP Finance}{Indonesian Students' Association Edmonton \\ {\footnotesize PERMIKA Edmonton}}{January 2019 -- December 2020}{}
  \begin{itemize}
    \item Helped restart the chapter after 2 years of inactivity due to lack of manpower.
    \item Manage finances and budgeting for events and fundraisers.
    \item Successfully budgeted a fundraisers that allowed us to operate comfortably.
    \item Secured funding in the form of grants and sponsorships that allowed our group to grow.
  \end{itemize}
  \divider\small

  \cvevent{Software Team Member}{Autonomous Robotic Vehicle Project}{May 2018 -- December 2019}{}
  \begin{itemize}
    \item Increased productivity by writing scripts that help automate various repetitive tasks, interfacing with Google App APIs, GitHub’s API and simulation scripts for ROS in C++ and Python.
    \item Improved test coverage for robot’s control systems by writing Google Test unit tests in C++.
    \item Updated toolchain migrated code from 5 year old tools, improving build times and allowing the Computer Vision team to use modern libraries.
    \item Updated legacy undocumented code in Mission Planning program to fix issues and add features.
  \end{itemize}
  \divider\small

\end{paracol}

\end{document}

