\documentclass[10pt,a4paper,ragged2e]{resume-openfont}

\geometry{left=1.25cm,right=1.25cm,top=1.5cm,bottom=1.5cm,columnsep=1.2cm}

\usepackage[utf8]{inputenc}
\usepackage{nag}
\usepackage{paracol}

\setmainfont{Lato}

\definecolor{Black}{HTML}{000000}
\definecolor{SlateGrey}{HTML}{2E2E2E}
\colorlet{heading}{Black}
\colorlet{accent}{Black}
\colorlet{emphasis}{SlateGrey}
\colorlet{body}{SlateGrey}

\renewcommand{\itemmarker}{\small\faAngleRight}
\renewcommand{\ratingmarker}{\faCircle}

\begin{document}
\name{Michael Kwok}
\personalinfo{
  \email{mkwok1@ualberta.ca}
  \github{n30phyte}
  \linkedin{michael-kwok00}
  \phone{587 937 5980}
}

\makecvheader
\columnratio{0.6}
\cvsection{Education}
\cvevent{BSc. Computer Engineering}{University of Alberta}{September 2017 -- April 2022}{Edmonton, AB, Canada}

\cvsection{Work Experience}
\cvevent{Software Engineer Intern}{Verdi Agriculture}{May 2021 -- August 2021}{}
\begin{itemize}
  \item Refactored over 6000 lines of backend code from ES5 JavaScript to modern TypeScript.
        %instead of "Overhauled backend code (around 7000 lines of code)" You would just say "Refactored ~(better to say >6,000 then ~7,000 imo) 7,000 lines of backend code..."
        %
        % \item Completely restructured backend to improve maintainability
        % \item Improved backend performance(?):
        % Before:
        % Mem: Max 194.2, Avg 85.4
        % Response times: Median 4ms Minimum 95th 59ms Maximum 95th 207ms
        % Max Requests per min: ~100RPM
        % After: TBD
  \item Designed new API endpoints for backend to minimize bandwidth usage and allow for better flexibility in the future.
  \item Implemented rigorous unit testing for backend code, from no coverage to 80\% test coverage on API endpoints.
  \item \textbf{Technologies:} JavaScript, TypeScript, Jest, MongoDB, GraphQL
\end{itemize}
\divider\small

\cvevent{Contract App Developer}{AdMore Lighting Inc}{August 2020 -- December 2020}{}
\begin{itemize}
  \item Developed a mobile app as an independent contractor, working closely with full time firmware developer.
  \item Improve existing app's performance and maintainability.
  \item Structure app to allow for automated testing with an external hardware jig in the future.
  \item \textbf{Technologies:} Dart, Flutter, Firebase, Protobuf, Bluetooth Low Energy
\end{itemize}

\cvsection{Projects}

\cvproject{rlox}{Aug 2021}{\href{https://github.com/n30phyte/rlox}{n30phyte/rlox}}
\begin{itemize}
  \item A JIT Compiler/interpreter written in Rust.
  \item \textbf{Technologies:} Rust
\end{itemize}
\divider\small

\cvproject{Rustracer}{Aug 2020}{\href{https://github.com/n30phyte/rustracer}{n30phyte/rustracer}}
\begin{itemize}
  \item A ray tracer written within 24 hours in Rust, with optimizations added over time.
  \item \textbf{Technologies:} Rust, Rayon
\end{itemize}
\divider\small

\cvproject{BudgetPacman (Hackathon Project)}{Jan 2020}{\href{https://github.com/loravocado/BudgetPacman}{loravocado/BudgetPacman}}
\begin{itemize}
  \item A project for HackED 2020 with the goal of making a version of tag that used a phone and GPS locations.
  \item Implemented a server using Socket.io and TypeScript, learning both technologies overnight.
  \item \textbf{Won:} Best Use of Google Cloud Platform APIs.
  \item \textbf{Technologies:} React Native, TypeScript, JavaScript
\end{itemize}
\divider\small

% \cvproject{AI Pong (Hackathon Project)}{Nov 2018}{\href{https://github.com/n30phyte/HackEDBeta2018}{n30phyte/HackEDBeta2018}}
% \begin{itemize}
%   \item Written during HackED Beta 2018, a hackathon hosted by the University of Alberta Computer Engineering Club.
%   \item First non-toy program in C++, using SFML to write a Pong clone to be controlled by a Python AI agent.
%   \item \textbf{Technologies:} C++, SFML, Python, Keras, TensorFlow
% \end{itemize}
% \divider\small

\cvsection{Technologies \& Programming Languages}

\cvsubsection{Proficient}

\cvtag{Python}
\cvtag{Git}
\cvtag{LaTeX}
\cvtag{CMake}
\cvtag{Dart}
\cvtag{C}
\cvtag{C++}
\cvtag{CUDA}
\cvtag{TypeScript}
\cvtag{JavaScript}

\cvsubsection{Familiar}

\cvtag{C\#}
\cvtag{Java}
\cvtag{SQL}
\cvtag{Rust}
\cvtag{VHDL}
\cvtag{React}
\cvtag{SFML}
\cvtag{MongoDB}
\cvtag{Firebase}
\cvtag{Ghidra}

\cvsection{Extra-Curriculars}

%% Combine last 2 points, explain how we got the money

% \cvevent{VP Finance}{Indonesian Students' Association Edmonton \\ {\footnotesize PERMIKA Edmonton}}{January 2019 -- December 2020}{}
% \begin{itemize}
%   \item Revived the group after 2 years of inactivity due to lack of manpower.
%   \item Increased our revenue by 800 dollars within a year.
%   \item This funding allowed our group to hold more events, increasing engagement from 5 to 6 people per event to more than 20.
% \end{itemize}
% \divider\small

\cvevent{Software Team Member}{Autonomous Robotic Vehicle Project}{May 2018 -- December 2019}{}
\begin{itemize}
  \item Reduce wasted time by writing scripts that help automate repetitive tasks, interfacing with Google App APIs, GitHub’s API and simulation scripts for ROS in C++ and Python.
  \item Improved test coverage for robot’s control systems by writing Google Test unit tests in C++.
  \item Updated technology stack, improving build times and opening access to updated CUDA libraries, increasing performance by 5 times.
\end{itemize}
\divider\small

\end{document}
